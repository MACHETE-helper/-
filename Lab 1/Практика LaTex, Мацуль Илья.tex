\documentclass{article}
\usepackage[utf8]{inputenc}
\usepackage[russian]{babel}
\usepackage{amsmath,amssymb,amsthm}
\usepackage{algorithm}
\usepackage[noend]{algpseudocode} 
\usepackage{graphicx}
\usepackage[T1,T2A]{fontenc}

\usepackage{fancyhdr}
\pagestyle{fancy}
\fancyhf{}
\renewcommand{\headrulewidth}{0pt}
\rhead{\textbf{\thepage}}
\setcounter{page}{299}
\chead{\textbf{Конечные цепные дроби}}
\lhead{\textbf{Раздел 13-3}}

\begin{document}
	

		\noindent где j наибольшее целое число, меньшее или равное (n-1)/2. Выведите этот результат. [Подсказка: аргументируйте индукцией, используя отношение $u_n = u_{n-1} + u_{n-2}$; отметим также, что ${m\choose k} = {m-1\choose k}+{m-1\choose k-1}$.]\\

\paragraph{12.} Докажем это для n$\geqslant$1,\\
\\
\indent (a) ${n\choose 1}u_1 + {n\choose 2}u_2 + {n\choose 3}u_3 + \ldots + {n\choose n}u_n = u_{2n};$\\
\\
\indent (b) $-{n\choose 1}u_1 + {n\choose 2}u_2 - {n\choose 3}u_3 + \ldots + (-1)^n{n\choose n}u_n = -u_{2n}.$

\subsection*{13.3 Конечные цепные дроби} В той части Liber Abaci, которая касается разложения дробей на единичные дроби, Фибоначчи ввёл своего рода "цепную дробь". Например, он использовал символ $\frac{1\;1\; 1}{3\; 4\; 5}$ в качестве аббревиатуры для

	$$\displaystyle\frac{1+\frac{1+\frac{1}{5}}{4}}{3}=\frac{1}{3} + \frac{1}{3\cdot4} + \frac{1}{3\cdot4\cdot5}$$


\noindent Современная практика, однако, заключается в том, чтобы записывать цепные дроби в нисходящем порядке,как с

$$2+\cfrac{1}{
	4+\cfrac{1}{
		1+\cfrac{1}{
			3+\cfrac{1}{2}}}}.$$

\noindent Многоуровневое выражение этого типа называется конечной простой цепной дробью. Чтобы поставить вопрос формально:\\\\
\noindent{\textsc{Определение 13-1.}Под конечной цепной дробью понимается дробь вида

	 $$a_0+\cfrac{1}{
	 	a_1+\cfrac{1}{
	 		a_2+\cfrac{1}{
	 			a_3+\cfrac{1}{
	 				\rotatebox{180}{$\ddots$}}}}}$$
 				\begin{align*}
 				&&&&\cfrac{1}{a_{n-1}+\cfrac{1}{a_n}},
 				\end{align*}
 				
 \newpage
 \chead{\textbf{Числа Фибоначчи и цепные дроби}}
 \rhead{\textbf{Раздел 13-3}}
 \lhead{\textbf{\thepage}}
 
 \noindent где $a_0,a_1,\cdots,a_n$ вещественные числа, все из которых, кроме возможно $a_0$ положительные. Числа $a_1,a_2,\cdots,a_n$ являются частичными знаменателями этой дроби. Такая дробь называется простой, если все $a_i$ целые числа.\\
 \indent Отдавая должное Фибоначчи, большинство учёных согласны с тем, что теория цепных дробей начинается с Рафаэля Бомбелли, последнего из великих Алгебраистов Италии эпохи Возрождения. В своей опере "L'Algebra" \\(1572),  Бомбелли попытался найти квадратные корни с помощью бесконечных цепных дробей - метод одновременно гениальный и новый. Он фактически доказал, что $\sqrt13$ можно выразить как цепную дробь

 	$$\sqrt13=3+\cfrac{4}{
 		6+\cfrac{4}{
 			6+\cfrac{4}{
 				6+\rotatebox{180}{$\ddots$}}}}.$$

 \noindent Интересно отметить, что Бомбелли был первым, кто популяризировал труд Диофанта на латинском Западе. Первоначально он намеревался перевести в Ватиканской библиотеке экземпляр "Арифметики" Диофанта (вероятно,тот же манускрипт,обнаруженный Региомонтаном), но, увлёкшись другими трудами, так и не закончил проект. Вместо этого он взял все задачи первых четырёх книг и воплотил их в алгебре, перемежая своими собственными задачами. Хотя Бомбелли не делал различий между этими задачами, он тем не менее признал, что свободно заимствовал их из арифметики.\\
 \indent Очевидно, что значение любой конечной простой цепной дроби всегда будет рациональным числом. Например, цепная дробь
 
 	$$3+\cfrac{1}{
 		4+\cfrac{1}{
 			1+\cfrac{1}{
 				4+\cfrac{1}{2}}}}$$
	 
её можно сократить до значения 170/53:

\begin{center}
	$3+\cfrac{1}{
		4+\cfrac{1}{
			1+\cfrac{1}{
				4+\cfrac{1}{2}}}}=3+\cfrac{1}{
		4+\cfrac{1}{
			1+\cfrac{2}{9}}}
	=3+\cfrac{1}{
		4+\cfrac{9}{11}}
	=3+\cfrac{11}{53}=\cfrac{170}{53}.$						
\end{center}
	
\newpage
\rhead{\textbf{\thepage}}
\chead{\textbf{Конечные цепные дроби}}
\lhead{\textbf{Раздел 13-3}}

\newtheorem*{theorem}{Теорема 13-5}
\begin{theorem}
	Любое рациональное число может быть записано как конечная простая цепная дробь.
\end{theorem}

\begin{proof}[Доказательство:]
	Пусть a/b, где b>0, любое рациональное число. Алгоритм Евклида для нахождения наибольшего общего делителя a и b дает нам следующие равенства\\
	
	\begin{center}
		\begin{align*}
		&a=ba_0 + r_1, &&0<r_1<b\\ 			
		&b=r_1a_1 + r_2, &&0<r_2<r_1\\
		&r_1=r_2a_2 + r_3, &&0<r_3<r_2\\
		&\vdots\\
		&r_{n-2}=r_{n-1}a_{n-1} + r_n, &&0<r_n<r_{n-1}\\
		&r_{n-1}=r_na_n + 0.\\		
	\end{align*}
\end{center}
	
Обратите внимание, что, так как каждый остаток $r_k$ положительное целое число, то $a_1,a_2,\cdots,a_n$ все положительные. Перепишите уравнения алгоритма следующим образом:\\

\begin{center}
	\begin{align*}
	a/b=&a_0 + r_1/b=a_0 + 1/(b/r_1),\\
	b/r_1=&a_1 + r_2/r_1=a_1 + 1/(r_1/r_2),\\
	r_1/r_2=&a_2 + r_3/r_2=a_2 + 1/(r_2/r_3)\\
	\vdots\\
	r_{n-1}/r_n&=a_n.			
	\end{align*}
\end{center}

Если исключить $b/r_1$ из первого равенства, то	получим

$$a/b=a_0 + 1/(b/r_1)=a_0+\cfrac{1}{
							a_1+\cfrac{1}{r_1/r_2}}.$$
						
В этом результате подставим значение $r_1/r_2$, заданное третьим равенством:

$$a/b=a_0+\cfrac{1}{
			a_1+\cfrac{1}{
				a_2+\cfrac{1}{r_2/r_3}}}.$$
			
\newpage
\chead{\textbf{Числа Фибоначчи и цепные дроби}}
\rhead{\textbf{Раздел 13-3}}
\lhead{\textbf{\thepage}}

\noindent Продолжая таким образом, мы можем пойти дальше, чтобы получить

 $$a/b=a_0+\cfrac{1}{
	a_1+\cfrac{1}{
		a_2+\cfrac{1}{
			a_3+
				\rotatebox{180}{$\ddots$}}}}$$
\begin{align*}
&&&&\cfrac{1}{a_{n-1}+\cfrac{1}{a_n}},
\end{align*}
	
тем самым заканчивая доказательство.	
	
\end{proof}

\indent Чтобы проиллюстрировать процесс, связанный с доказательством теоремы 13-5, представим 19/51 как цепную дробь. Применение алгоритма Евклида к целым числам 19 и 51 даёт уравнения

\begin{center}
	\begin{align*}
		51&=2\cdot+13\qquad &&\text{или}& \qquad 51/19&=2+13/19,\\
		19&=1\cdot13+6\qquad &&\text{или}& \qquad 19/13&=1+6/13,\\
		13&=2\cdot6+1\qquad &&\text{или}& \qquad 13/6&=2+1/6,\\
		6&=6\cdot1+0\qquad &&\text{или}& \qquad 6/6&=1.\\
	\end{align*}
\end{center}

Производя соответствующие замены, видно, что

\begin{center}
	\begin{align*}
	\frac{19}{51}=\frac{1}{51/19}&=\frac{1}{2+\frac{13}{19}}
	=\frac{1}{2+\frac{1}{\frac{19}{13}}}
	=\cfrac{1}{2+\cfrac{1}{1+\cfrac{6}{13}}}\\
	&=\cfrac{1}{2+\cfrac{1}{1+\cfrac{1}{\frac{13}{6}}}}\\
	&=\cfrac{1}{2+\cfrac{1}{1+\cfrac{1}{2+\frac{1}{6}}}}\\
	\end{align*}
\end{center}

\noindent что является расширением цепной дроби для 19/51.
	
\end{document}


